\documentclass[a4,11pt]{article}
\topmargin = 2pt
\textwidth = 450pt
\textheight = 640pt
\marginparwidth = 90pt
\evensidemargin = 54pt

\usepackage{xcolor}
\usepackage{enumerate}
\usepackage{multicol}
\usepackage[showframe=false]{geometry}
\usepackage{changepage}

\usepackage{listings}
\lstdefinestyle{Bash}
{language=bash,
keywordstyle=\color{blue},
basicstyle=\ttfamily,
morekeywords={peter@kbpet},
alsoletter={:~\$},
morekeywords=[2]{ak2@ak2-XPS-L501X:},
keywordstyle=[2]{\color{red}},
literate={\$}{{\textcolor{red}{\$}}}1 
         {:}{{\textcolor{red}{:}}}1
         {~}{{\textcolor{red}{\textasciitilde}}}1,
}

\begin{document}

\title{CS342 \\
	Report\\
	Assignment 3: Wireshark}
\author{Anil Kag\\
	   10010111\\
	   a.kag@iitg.ernet.in}
\date{March 20, 2013}
\maketitle

\section{Part 1\\
	  Basics}
\begin{enumerate}
 %prb 1%
 \item Protocols which appear in the protocol column in the unfiltered packet-listing
    \begin{multicols}{3}
      \begin{enumerate}
      \item ARP  	\item DHCiPv6
      \item DNS 	\item Ethernet 
      \item ICMP      	\item HTTP
      \item LLC        \item LLMNR
      \item SSDP       \item STP 
      \item TCP       	\item UDP
      \end{enumerate}
    \end{multicols}
    
    %prb 2%
    \item  Packet short details are as follows \\
      \begin{adjustwidth}{-2cm}{}
      \begin{tabular}{|r|l|l|l|l|l| p{4cm} |}
       \hline
       no&	time&			source&			destination&		proto&	len&	info \\
       \hline
       310& 	17:00:18.753711& 	172.16.27.59& 		202.141.80.22& 		HTTP& 	867& 	GET http://www.google.co.in/ HTTP/1.1 \\
       391& 	17:00:18.903545&	202.141.80.22&		172.16.27.59&		HTTP&	66&	HTTP/1.0 200 OK  (text/html)\\
       \hline
      \end{tabular}
      \end{adjustwidth}

      Time taken = 0.903545-.753711 sec = 0.149834 sec

      %prb 3%
      \item IP of google cannot be determined by looking at the above two packets because the proxy server(202.141.80.22) handles
	      the connection part between my host \& google.com \\

	    Host ip $\Rightarrow$ 172.16.27.59
\end{enumerate}
	  
\section{Part 2\\
	  Ethernet}
	 
    \textit{Reference packets used for solving this part}\\
      \begin{adjustwidth}{-2cm}{}
      \begin{tabular}{|r|l|l|l|l|l| p{4cm} |}
       \hline
       no&	time&			source&			destination&		proto&	len&	info \\
       \hline
       113& 	17:47:02.684944& 	172.16.27.59& 		202.141.80.22& 		HTTP& 	668& 	GET http://www.faqs.org/rfcs/rfc826.html HTTP/1.1 \\
       391& 	17:47:03.771542&	202.141.80.22&		172.16.27.59&		HTTP&	3935&	HTTP/1.0 200 OK  (text/html)\\
       \hline
      \end{tabular}
      \end{adjustwidth}
	  
\begin{enumerate}
  %prb 1%
  \item HTTP GET message's ethernet header \\
      Ethernet II, Src: Pegatron\_b3:05:c4 (38:60:77:b3:05:c4), Dst: Cisco\_9d:70:00 (00:24:f9:9d:70:00) \\

      Ethernet address of your computer	38:60:77:b3:05:c4	\\

   %prb 2%
  \item Destination Ethernet address 		00:24:f9:9d:70:00	\\

  It's not the ethernet address of the RFC website. \\
  It's actually the ethernet address of the next hop for reaching the destination in my computer's routing table. \\

  You can actually check the IP for the device having destination ethernet address as this by 
  running 'arp -n' on your linux machine \& check the IP corresponding to this ethernet address on my machine.

  \begin{adjustwidth}{-2cm}{}
  \begin{lstlisting}[style=bash]
    $ arp -n
    Address     	HWtype  HWaddress           Flags Mask   Iface
    172.16.27.68       ether   f0:4d:a2:4f:15:6d   C            eth0
    172.16.24.254      ether   00:24:f9:9d:70:00   C            eth0
  \end{lstlisting}
  \end{adjustwidth}
  
  $\Rightarrow$ it's ethernet address of 172.16.24.254

  %*** incomplete prb3%
  \textit{Incomplete}
  \item Type field value = 0x0800 $\Rightarrow$ IP packet
	** What about the flags?

  %*** prb 4 incomplete%
  \item Ethernet Header Contents
  \textit{Incomplete}
  \begin{adjustwidth}{-2cm}{}
  \begin{lstlisting}[style=bash]
    0000  00 24 f9 9d 70 00 38 60  77 b3 05 c4 08 00 45 00   .$..p.8` w.....E.
    0010  02 8e 56 bd 40 00 40 06  ff bd ac 10 1b 3b ca 8d   ..V.@.@. .....;..
    0020  50 16 c9 6a 0c 38 94 40  d2 f1 66 42 57 69 80 18   P..j.8.@ ..fBWi..
    0030  00 e5 28 7f 00 00 01 01  08 0a 00 9e 84 4b 15 b1   ..(..... .....K..
    0040  50 ee 47 45 54 20 68 74  74 70 3a 2f 2f 77 77 77   P.GET ht tp://www
  \end{lstlisting}
  \end{adjustwidth}
  
  ASCII letter 'G' starts on line 5 with base $0x0040$ \& offset $0x0003$ \\
  $\Rightarrow$ position (in bytes) from the start of Ethernet Frame $= 4*16+3 = 67$ ($0x0010 = 16$ in decimal)

  %prb 5%
  \item HTTP Response \\
  Ethernet II, Src: Cisco\_9d:70:00 (00:24:f9:9d:70:00), Dst: Pegatron\_b3:05:c4 (38:60:77:b3:05:c4) \\
  src = 00:24:f9:9d:70:00 $\Rightarrow$ hop just before my computer in the path from website to my computer \\

  %prb 6%
  \item dst = 38:60:77:b3:05:c4 $\Rightarrow$ my computer (you can verify via ifconfig \& look at the hwaddress for the eth0 interface)

  \begin{adjustwidth}{-2cm}{}
  \begin{lstlisting}[style=bash]
    $ ifconfig
    eth0      Link encap:Ethernet  HWaddr 38:60:77:b3:05:c4  
	      inet addr:172.16.27.59  Bcast:172.16.27.255  Mask:255.255.252.0
	      inet6 addr: fe80::3a60:77ff:feb3:5c4/64 Scope:Link
	      UP BROADCAST RUNNING MULTICAST  MTU:1500  Metric:1
	      RX packets:8008125 errors:0 dropped:4342 overruns:0 frame:0
	      TX packets:173669 errors:0 dropped:0 overruns:0 carrier:0
	      collisions:0 txqueuelen:1000 
	      RX bytes:1129843477 (1.1 GB)  TX bytes:20988447 (20.9 MB)
	      Interrupt:43 Base address:0xe000 

    lo        Link encap:Local Loopback  
	      inet addr:127.0.0.1  Mask:255.0.0.0
	      inet6 addr: ::1/128 Scope:Host
	      UP LOOPBACK RUNNING  MTU:16436  Metric:1
	      RX packets:21789 errors:0 dropped:0 overruns:0 frame:0
	      TX packets:21789 errors:0 dropped:0 overruns:0 carrier:0
	      collisions:0 txqueuelen:0 
	      RX bytes:1986523 (1.9 MB)  TX bytes:1986523 (1.9 MB)
  \end{lstlisting}
  \end{adjustwidth}

  %*** prb 7 incomplete%
  \textit{Incomplete}
  \item Type field == 0x0800 $\Rightarrow$ IP \\
    ** what about flags?

\end{enumerate}

	  
\section{Part 3\\
	  IP}

  \begin{enumerate}
   \item Internet Protocol Version 4, Src: 172.16.27.59 (172.16.27.59), Dst: 202.141.80.21 (202.141.80.21) \\
	 IP my computer = 172.16.27.59

    \item Protocol: UDP (17)

    \item Internet Header Length = 20bytes \\
    (if only looking at packet, value given in IHLen is 5 $\Rightarrow 5*32$ bits = $5*4$bytes = 20 bytes) \\

    ** Is it correct? \\
    Total Length = 56bytes = Header Length + IP Payload Length \\
    $\Rightarrow$ IP Payload Length = 56bytes - 20bytes = 36bytes \\
    (as Header length = 20bytes)

    \item Flags: 0x00 \\
      0... .... = Reserved bit:   Not set \\
      .0.. .... = Don't fragment: Not set \\
      ..0. .... = More fragments: Not set \\
  Fragment offset: 0 \\
  Since fragment offset is 0 \& no more fragments are going to come \\
  $Rightarrow$ no fragmentation

  \item Identification \& Checksum always change while going from one packet to other \\
   ** Will TTL come here \\
  For each three packet TTL will be fixed \& after that it'll be incremented by 1

  \item const fields $\Rightarrow$ Version, Header Length(?), Protocol = UDP, Src \& Dest IP \\
	field may change $Rightarrow$ TTL, More Fragments, Total Length, Fragment Offset \\
	Which fields must change \& why? \\

    ** More robust answer

  \item Identification \& TTL values do not remain same\\
    ** why?

  \end{enumerate}
	  
\section{Part 4\\
	  UDP}
  \begin{enumerate}
   \item 
   %\begin{multicols}{2}
    \begin{enumerate}
      \item source port
      \item destination port
      \item length
      \item checksum
    \end{enumerate}
  %\end{multicols}

  \item Each Field is 4bytes(16 bits) long

  \item Select the DNS query portion \& it expands over 46bytes which is equal to the length given in the UDP packet \\
      $\Rightarrow$ length in UDP packet refers to the actual data length

  \item Protocol Number =  17(decimal), 0x11(hexadecimal)

  %Format it nicely %
  \item 
  Request		"39","23:43:44.573062","172.16.27.59","202.141.80.9","DNS","80","Yes","Standard query A jampui.iitg.ernet.in" \\
		  Internet Protocol Version 4, Src: 172.16.27.59 (172.16.27.59), Dst: 202.141.80.9 (202.141.80.9) \\
		  User Datagram Protocol, Src Port: 56060 (56060), Dst Port: domain (53)		\\

  Response	"42","23:43:44.573688","202.141.80.9","172.16.27.59","DNS","171","Yes","Standard query response A 202.141.80.21" \\
		  Internet Protocol Version 4, Src: 202.141.80.9 (202.141.80.9), Dst: 172.16.27.59 (172.16.27.59) \\
		  User Datagram Protocol, Src Port: domain (53), Dst Port: 56060 (56060) \\

  Source port in one becomes the destination in other \& vice-versa
  \end{enumerate}

	  
\end{document}